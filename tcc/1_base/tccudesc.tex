% The original is from UDESC
% Lambd0x (Nicholas P. Lane) changed it
% Now I am using it

\NeedsTeXFormat{LaTeX2e}

\documentclass[a4paper,12pt]{monografia}
\usepackage{amsmath, amsthm, amsfonts, amssymb}
\usepackage[english, brazilian]{babel}
\usepackage[titletoc]{appendix}
\usepackage[T1]{fontenc}
\usepackage[latin1]{inputenc}
\usepackage[mathcal]{eucal}
\usepackage[alf]{abntcite}
\usepackage[font=small,labelfont=bf,tableposition=top]{caption}
\usepackage{subfigure}
\usepackage{isoaccent}
\usepackage{textcase}
\usepackage{multirow}
\usepackage{latexsym}
\usepackage{graphicx}
\usepackage{listings}
\usepackage{acronym}

\usepackage{enumitem}
\usepackage{url}

\usepackage{bm}

\setlist[description]{leftmargin=\parindent,labelindent=\parindent}

\lstset{basicstyle=\tiny, language=C++}

\makeindex


\begin{document}

\titulo{Proposta de um algoritmo hibrido para a solu��o do Problema de Escalonamento de Tripula��o}
\autor{Renan Samuel da Silva}
\nome{Samuel}
\ultimonome{da Silva}

\bacharelado \curso{Ci\^encia da Computa\c{c}\~ao} \mes{Junho} \ano{2016}
\data{\today}
\cidade{Joinville}

\instituicao{Universidade do Estado de Santa Catarina}
\sigla{UDESC} \unidadeacademica{Centro de Ci\^encias Tecnol\'ogicas}

\orientador{Omir Correia Alves Junior}
\examinadorum{Cristiano Damiani Vasconcellos}
\examinadordois{Cl�udio C�sar de S�}

\ttorientador{Doutor}
\ttexaminadorum{Doutor}
\ttexaminadordois{Doutor}

\maketitle

%\agradecimento{Agradecimentos}
  %Agrade�o a minha perna, por ter me levado a muitos lugares. Aos meus bra�os, que sempre estiveram ao meu lado e aos meus dedos, o qual eu sempre pude contar.

\newpage

\begin{epigrafe}

  ``We must know - We will know'' - David Hilbert

\end{epigrafe}

\resumo{Resumo}
A \ac{TA} � uma �rea do conhecimento utilizada para identificar os recursos para proporcionar ou ampliar habilidades de pessoas com defici�ncia, incapacidades ou com mobilidade reduzida. 
%
Neste trabalho � realizada uma pesquisa referenciada sobre pessoas com defici�ncias, mais especificamente pessoas que possuem \ac{PC}, seguindo para uma pesquisa aplicada na qual � definida a especifica��o e o desenvolvimento de um software de Comunica��o Alternativa Ampliada. 
%
No conjunto de portadores de \ac{PC} o trabalho trata em especial das pessoas que possuem habilidades locomotoras limitadas em conjunto com dificuldades na fala. 
%
A especifica��o de solu��o alternativa apresentada neste trabalho, possibilita que essas pessoas se comuniquem com os seus terapeutas com objetivo de estimular sua cogni��o. 
%
O trabalho est� organizado em tr�s etapas: o contexto, que diz respeito as pessoas com defici�ncia; as iniciativas de \ac{TA} e as suas classifica��es, a especifica��o do problema que � abordada no trabalho; e a especifica��o da proposta deste trabalho.
%
Contudo, �s iniciativas em \ac{TA} tendem a possuir alguns dos seus aspectos regionilizados em fun��o das demandas locais.
%
Neste sentido, esse trabalho atuou junto a \ac{ADEJ} para identificar esse contexto regionalizado e o Grupo Assistiva da \ac{UDESC}.  


\noindent Palavras-chaves: Programa��o linear inteira, otimiza��o combinat�ria, cobertura de conjunto, agendamento de tripula��o, gera��o de colunas.

\resumo{Abstract}
In this work we approach the crew scheduling problem (CSP), a $\mathcal{NP}$-hard problem.
To deal with it's intractability, we did a research in search of exact methods capable of
working with the large amount of variable that the problem has. We decided to use the
column generation method to obtain an exact solution to the problem and to use
one or more (meta) heuristics to reduce the processing time.


\noindent Keywords: Integer linear programming, combinatorial optimization, set covering, crew scheduling, column generation.

\listoffigures

\listoftables

\chapter*{Lista de Siglas e Abreviaturas}
\addcontentsline{toc}{chapter}{Lista de Siglas e Abreviaturas}
\begin{acronym}

\acro{SPP}[SPP]{Set Partitioning problem}
\acro{SCP}[SCP]{Set Covering problem}
\acro{CSP}[CSP]{Crew Scheduling problem}
\acro{PLI}[PLI]{Programa��o Linear Inteira}

\end{acronym}


\tableofcontents

\pagestyle{ruledheader}

\chapter{Introdu��o}
\label{cap:introducao1}

Cada vez mais as empresas buscam otimizar suas atividades, de modo a reduzir o custo e maximizar o lucro final.
As empresas do segmento de transporte urbano de �nibus deparam-se com desafios que precisam ser abordados
a fim de viabilizar o neg�cio, dentre as quais pode-se citar: renova��o da frota de ve�culos;
realiza��o da manuten��o preventiva da frota; disponibiliza��o de quais hor�rios do servi�o de �nibus;
identifica��o as rotas de servi�o de transporte; aloca��o de funcion�rios; tratamento de imprevistos
temporais e clim�ticos, dentre outras. Estes questionamentos podem ser resolvidos e otimizados com a utiliza��o de recursos
de pesquisa operacional~\cite{op_survey}.

O processo de planejamento operacional de uma empresa de �nibus � dividido em diversas etapas, segundo descrito
por~\cite{de2011algoritmo} e~\cite{qiao2010algorithm}. Figura~\ref{fig_etapas} cont�m uma representa��o gr�fica
da intera��o entre as etapas e a ordem na qual elas s�o executadas.

{
    \centering
    \includegraphics[height=0.3\textheight]{../figuras/etapas.pdf}
    \captionof{figure}{Etapas do planejamento, adaptado de~\cite{de2011algoritmo}}
    \label{fig_etapas}
}

A primeira etapa consiste em determinar quais locais devem ser atendidos pelo servi�o de transporte; Determinar qual
a melhor rota para cobrir estes locais; Deve ainda escolher a melhor localiza��o para as esta��es de �nibus ao longo
do percurso do ve�culo. A segunda etapa consiste em, determinar os hor�rios na qual os
�nibus devem partir dos terminais, de modo a atender da forma mais eficiente poss�vel a demanda. A terceira etapa
consiste em determinar quais ve�culos partem de quais esta��es; Por quais pontos ele deve passar e para onde que
ele deve retornar, formando assim uma jornada. Esta etapa deve cobrir todas as tarefas e minimizar o custo operacional
A quarta e �ltima etapa deve associar membros da tripula��o(aloca��o) com jornadas, de modo que cada jornada tenha a tripula��o necess�ria,
sendo que cada membro da tripula��o deve ter todos os seus direitos respeitados, tanto os previstos por lei quanto os contratuais.

Especificamente, o problema de aloca��o de funcion�rios ou tripula��o, em ingl�s \textit{crew scheduling problem} (CSP),
consiste em escolher grupos de funcion�rios que devem realizar tarefas durante um dado per�odo de tempo \cite{Bergh}. O processo
de escolha est� sujeito � restri��es como: Regras definidas pela empresa; leis trabalhistas; sindicais
ou prefer�ncias estipuladas pela pr�pria tripula��o. Dado um conjunto de restri��es, deve-se encontrar as designa��es que reduzem
ao m�ximo o custo da opera��o. O CSP assume que j� tenha sido determinado os percursos onde a tripula��o ir� trabalhar, assim como quais ve�culos
s�o utilizados, quais os hor�rios de partida e chegada e os pontos de troca.

A defini��o de termos relacionados � problem�tica do CSP faz-se necess�ria e s�o apresentadas a seguir: uma \textbf{viagem} ou \textbf{tarefa} consiste
no deslocamento entre dois pontos pr�-determinados com hor�rios de partida e chegada j� determinados. Uma \textbf{jornada}
consiste na sequencia de viagens realizadas por uma dada tripula��o durante o seu turno de trabalho. Conhecendo-se todas as poss�veis
jornadas v�lidas sobre o conjunto de regras pr�-estabelecidos, a solu��o do CSP consiste em escolher quais jornadas cobrem todas
as viagens que devem ser realizadas~\cite{Bergh}. Ou seja, dado o conjunto de jornadas v�lidas, escolhe-se um subconjunto que cubra todas
as viagens pelo menos uma vez, minimizando assim o custo final.

O problema do CSP � referenciado na literatura como sendo um problema de \textbf{cobertura de conjuntos}, ou \textit{set covering problem}
(SCP)~\cite{nemhauser1988integer}. Dado um conjunto de linhas a serem cobertas, e um conjunto de colunas com custos
associados que cobrem as linhas, escolhe-se o subconjunto de colunas que cubra todas as linhas minimizando o custo final. Como o SCP
define que cada linha deve ser coberta pelo menos uma vez, isso implica que se o CSP for modelado como uma inst�ncia de SCP, pode ocorrer
de mais de um tripulante ser designado para a mesma viagem. Isso na pr�tica consiste no fato da tripula��o ir de carona at� o in�cio de outra viagem,
por exemplo.

Outra forma de modelar-se o CSP consiste em determinar qual tripula��o deve ser associado a uma viagem. Esta restri��o
transforma o problema de SCP em um problema de \textbf{particionamento de conjuntos}, \textit{set partitioning problem} (SPP)~\cite{garfinkel1969set}.
A principal diferen�a entre o SCP e o SPP consiste na impossibilidade de ter-se uma mesma linha coberta por mais de uma coluna.
No contexto do CSP, isto implica em n�o ter-se uma tarefa sendo coberta por mais de uma jornada. Devido a restri��o de n�o
sobreposi��o, um CSP modelado como um SPP pode ser significativamente mais complexo de resolver, por�m, mais interessante do
ponto de vista pr�tico, j� que diminui a ociosidade dos tripulantes. Sabe-se que tanto o SPP quanto o SCP s�o problemas
NP-Hard~\cite{karp1972}.

Um dos primeiros algoritmos propostos para o SCP consiste em uma heur�stica gulosa proposta por~\cite{chvatal1979greedy}.
A cada passo o algoritmo escolhe a coluna que cobre o maior n�mero poss�vel de linhas. O algoritmo
� na pr�tica r�pido, por�m tende a n�o gerar solu��es t�o boas quanto outros algoritmos modernos.~\cite{balas1980set} prop�s
um algoritmo baseado em \textit{branch and bound} e que utiliza heur�sticas duais.
%Este algoritmo (com o poder computacional dispon�vel na data da publica��o do artigo) foi capaz de resolver inst�ncias de dimens�es at� 200 $\times$ 2000.
~\cite{beasley1987algorithm} melhorou o algoritmo proposto por Balas e Ho, utilizando relaxa��es lagrangianas e remo��o de linhas e colunas.
%Este algoritmo, segundo o autor, chegou a marca de problemas de dimens�es $400 \times 4000$.
~\cite{fisher1990optimal} utilizou \textit{branch and bound} com
diversas heur�sticas duais, para assim encontrar o limite superior de otimalidade do problema. Trabalhos como~\cite{de2011algoritmo}
e~\cite{dos2008metodo} utilizam solu��es h�bridas para a solu��o do problema.~\cite{ceria1997},~\cite{caprara2000algorithms},
~\cite{ernst2004staff} e~\cite{van2013personnel} apresentam um estudo mais detalhado sobre os algoritmos utilizados para resolver o
SCP, SPP e CSP.

%Conhecendo-se todas as poss�veis jornadas de uma inst�ncia de CSP, pode-se modela-lo utilizando programa��o linear inteira (PLI).
Diversas modelagens para o CSP s�o poss�veis,~\cite{beasley1987algorithm} utilizou uma modelagem baseada em m�ltiplos caminhos m�nimos.
� poss�vel ainda utilizar SCP e SPP para modelar o CSP. A modelagem utilizando-se
o SPP e o SCP � apresentada em~\eqref{s_pp} e~\eqref{scp}, respectivamente. O conjunto de todas as jornadas poss�veis est� codificado
na matriz $A$. O vetor $J$ corresponde a todas as jornadas, e o vetor $T$ a todas as tarefas. A vari�vel $a_{tj}$ � $1$ se a tarefa
$t$ � coberta pela jornada $j$ e $0$ caso contr�rio. A restri��o~\eqref{s_pp2} garante que cada tarefa $t \in T$ � coberta exatamente
uma �nica vez pelas jornadas selecionadas.

A restri��o~\eqref{scp2} funciona de modo an�logo, por�m, a restri��o deixa de ser uma igualdade
e passa a ser uma desigualdade, fazendo com que cada tarefa seja coberta pelo menos uma vez. O vetor $X$ determina que jornadas
ser�o utilizadas. Se $x_j = 1$, ent�o a $j-$�sima � utilizada, caso contr�rio $x_j = 0$. As restri��es~\eqref{s_pp3} e~\eqref{scp3}
garantem que a var�vel de decis�o $x$ possuir� um valor v�lido.

Dado o problema em quest�o, o tamanho da matriz � tipicamente grande, e para casos pr�ticos
� muitas vezes invi�vel de trat�-la. O n�mero de jornadas poss�veis cresce exponencialmente, de modo que a enumera��o de todas as
jornadas poss�veis n�o � vi�vel.
\cite{vance1993crew} reportou que para uma inst�ncia de CSP com 253 tarefas, mais de 5 milh�es de jornadas
s�o poss�veis. De todas as jornadas poss�veis, apenas algumas s�o de fato utilizadas na solu��o final.
Portanto, � interessante que apenas jornadas que podem vir a ser �teis sejam utilizadas de fato.
%A programa��o linear inteira disp�e-se um recurso
%capaz de sequencialmente reduzir o n�mero de colunas a serem utilizados, denominado de \textbf{gera��o de colunas}~\cite{desaulniers2006column}.
A fim de reduzir o n�mero de colunas, o m�todo de gera��o de colunas pode ser utilizado
para resolver o problema~\cite{desaulniers2006column}.

\begin{align}
    \label{s_pp} \text{min} \: \sum_{j \in J} c_j x_j \\
    \label{s_pp2} \sum_{j \in J} a_{tj} x_j = 1, \forall t \in T \\
    \label{s_pp3} x_j \in \{0, 1\}, \forall j \in J
\end{align}

\begin{align}
    \label{scp} \text{min} \: \sum_{j \in J} c_j x_j \\
    \label{scp2} \sum_{j \in J} a_{tj} x_j \ge 1, \forall t \in T \\
    \label{scp3} x_j \in \{0, 1\}, \forall j \in J
\end{align}

O m�todo de gera��o de colunas consiste decompor um dado problema com um grande n�mero de vari�veis em
dois problemas menores: O \textbf{problema mestre}, que � derivado do problema original, por�m, com um conjunto reduzido de vari�veis,
e o \textbf{sub-problema},
que � utilizado para identificar quais vari�veis s�o necess�rias para obter-se a solu��o �tima do problema original. O processo
de solu��o com gera��o de colunas consiste em criar-se um problema mestre, otimizar a sua relaxa��o linear, e utilizar
as var�veis duais do problema mestre juntamente com o sub-problema,  para identificar se o problema mestre necessita da gera��o
de novas colunas ou se ele corresponde ao �timo do problema original~\cite{dos2008metodo}.

Al�m do m�todo de gera��o de colunas, outros m�todos tamb�m s�o utilizados na literatura, tais como: algoritmos gen�ticos~\cite{doalgoritmos},
\textit{simulated annealing}~\cite{hanafi2014hybrid},
\textit{particle swarm optimization}~\cite{limlawan2014hybrid}
\textit{multi-start randomized heuristic}~\cite{de2016multi},
e \textit{ant colony optimization}~\cite{deng2011ant}.
Na literatura pesquisada at� o momento, identificou-se que v�rios autores a fim de resolver o CSP de forma eficiente
(que apresentam um melhor desempenho computacional), utilizaram o m�todo de gera��o de colunas em conjunto com outros m�todos,
podendo ser heur�sticos ou h�bridos.

Dada a pesquisa que realizou-se, identificou-se que m�todos h�bridos exatos, apesar de seu custo computacional relativamente grande,
tem um potencial de oferecer a solu��o �tima para as inst�ncias do CSP, em um tempo menor do que um algoritmo puramente exato.

%Dada a pesquisa que realizou-se, n�o foi poss�vel concluir que identificou-se um m�todo (h�brido ou heur�stico)
%que apresenta o melhor desempenho computacional dentre todos os m�todos que foram propostos. Especialmente considerando
%que o CSP apresenta diversas modelagens que variam entre si, de modo que o bom desempenho para um conjunto de modelos
%n�o transfere-se para outro.

Dado o exposto acima, este trabalho tem o objetivo de especificar e propor um m�todo para a solu��o do CSP que, a princ�pio, seja capaz de prover
solu��es de modo exato, e preferencialmente mais eficientes, considerando os algoritmos pesquisados na literatura. O algoritmo proposto ser� aplicado
� inst�ncias dispon�veis na literatura.

No cap�tulo 2 � apresentado a fundamenta��o teoria necess�ria para o estudo do algoritmo proposto. Define-se programa��o linear, programa��o linear
inteira e � apresentado os seus respetivos m�todos de solu��o.

No cap�tulo 3 � apresentado a formula��o do CSP e sua respectiva modelagem utilizando-se gera��o de colunas, onde � definido tamb�m o problema mestre
e o subproblema.

No cap�tulo 4 apresenta-se a proposta para este trabalho. Discute-se as ferramentas que ser�o utilizadas e os m�todos considerados para resolver o subproblema.

\setlength\abovedisplayskip{0pt}
\setlength\belowdisplayskip{0pt}
\setlength\abovedisplayshortskip{0pt}
\setlength\belowdisplayshortskip{0pt}

\chapter{Fundamenta��o Te�rica}
Neste capitulo s�o apresentados os conceitos b�sicos e a fundamenta��o te�rica necess�ria para
o entendimento e abordagem ao problema do escalonamento de tripula��es.

\section{Programa��o Linear}

A programa��o linear consiste na modelagem e solu��o de problemas descritos com uma fun��o
objetivo linear sujeita a m�ltiplas restri��es lineares. A forma gen�rica de um problema de programa��o
linear �:

\begin{align} \label{funcao_obj}
    \text{maximizar} \: z &= \sum_{j=1}^{p} c_jx_j,
    \intertext{sujeito a:}
    \label{restricoes} \sum_{j=1}^{p} a_{ij} x_j \leq b_i, i &= 1, 2, \ldots, q \\
    \label{restricoes_triviais} x_j \geq 0, j &= 1, 2, \ldots, p,
\end{align}

onde $c_j, a_{ij}$ e $b_i$ s�o n�meros reais que definem o problema e $x_j$ para $j=1, 2, \ldots, p$ s�o as vari�veis
de decis�o. A fun��o~\eqref{funcao_obj} � denominada de fun��o objetivo, a qual deve ser maximizada. Note que
o m�ximo de $f(x)$ � o m�nimo de $-f(x)$ com o sinal oposto (max $f(x) = - \text{min} f(x)$. As inequa��es
em~\eqref{restricoes} representam um conjunto de $q$ restri��es lineares que restringem o espa�o de busca para um
poliedro convexo. O m�ximo da fun��o deve estar contido dentro deste poliedro. As desigualdades em~\eqref{restricoes_triviais}
s�o denominadas de restri��es triviais ou de n�o negatividade.

Cada restri��o em~\eqref{restricoes} pode ser convertida para restri��o de de igualdade utilizando-se
uma vari�vel extra, denominada de vari�vel de folga. A utiliza��o d�-se por:

\[
\label{restricoes} \sum_{j=1}^{p} a_{ij} x_j \leq b_i \Leftrightarrow
  \begin{cases}
    \Sigma_{j=1}^{p} a_{ij} x_j + x_{p+i} = b_i\\
    x_{p+i} \geq 0
  \end{cases}
\]

� poss�vel ainda utilizar duas igualdades para representar uma igualdade:

\[
\label{restricoes} \sum_{j=1}^{p} a_{ij} x_j = b_i \Leftrightarrow
  \begin{cases}
    \Sigma_{j=1}^{p} a_{ij} x_j \leq b_i\\
    \Sigma_{j=1}^{p} a_{ij} x_j \geq b_i
  \end{cases}
\]

Para um problema qualquer de programa��o linear com restri��es de desigualdades e igualdades, sempre � poss�vel
reestrutura-lo atrav�s da adi��o de vari�veis de folga para que o problema passe a ter apenas igualdades. Portanto
todo e qualquer problema de programa��o linear pode ser expresso como:

\begin{align}
    \label{fo} \text{maximizar} \: z &= \sum_{j=1}^{n} c_jx_j
    \intertext{sujeito a:}
    \label{res} \sum_{j=1}^{n} a_{ij} x_j \leq b_i, i &= 1, 2, \ldots, m \\
    \label{res_t} x_j \geq 0, j &= 1, 2, \ldots, n
    \intertext{que � equivalente a:}
    \label{fo2} \text{maximizar} \: z &= cx
    \intertext{sujeito a:}
    \label{res2} Ax &= b \\
    \label{res_t2} x &\geq 0,
\end{align}

onde $c^T \in \mathbb{R}^n$, $x \in \mathbb{R}^n$, $b \in \mathbb{R}^m$, $A \in \mathbb{R}^{m \times n}$ e $a_j \in \mathbb{R}^m$.

Baseado nas restri��es~\eqref{res2} e~\eqref{res_t2}, pode-se descrever o problema como encontrar $x \in X$, onde $X = \{ x \in \mathbb{R}^n | Ax = b, x \geq 0 \}$,
tal que $cx$ seja maximizado. Ao conjunto $X$ d�-se o nome de regi�o factivel ou espa�o de busca. Para um $x \in X$ qualquer, diz-se que $x$ � uma solu��o fact�vel.
Para um $x^* \in X$, se $cx^* \geq cx \forall x \in X$, ent�o $x^*$ � a solu��o �tima do problema. O espa�o de busca para um problema de programa��o linear � sempre
um poliedro convexo, se interpretado geometricamente, e a solu��o �tima para o mesmo sempre est� em um v�rtice do poliedro.

\subsection{M�todos de solu��o}

Considerando o conjunto $X$ descrito na se��o anterior, o objetivo de efetuar a modelagem de um problema utilizando-se programa��o linear � utilizar recursos
m�tematicos e algor�tmicos para resolver o modelo, e por conseguinte o problema original. Esta se��o apresenta os tr�s principais m�todos encontrados na literatura durante
o desenvolvimento deste trabalho: o m�todo simplex; o m�todo dos elipsoides; o m�todo do ponto interior.

%\subsubsection{Simplex}

O m�todo Simplex~\cite{dantzig1990origins} foi apresentado em 1947 por George B. Dantzig com o objetivo de resolver o problema de programa��o linear. O m�todo consiste em encontrar uma solu��o
fact�vel ao problema, e iterativamente mover para uma solu��o melhor ou igual que a atual. Considerando que a solu��o est� e um v�rtice do poliedro, � necesses�rio
apenas explorar os v�rtices. Tendo em vista que existe um n�mero finito de vertices para um poliedro descrito por um conjunto finito de restri��es lineares (que
geometricamente correspondem a hiperplanos), fica claro que o simplex converge em um n�mero finito de passos. Apesar de que na m�dia o simplex
resolve o problema em um n�mero polinomial de passos, em 1972 foi apresentado uma prova de que o m�todo simplex no seu pior caso � exponencial~\cite{klee1972good}.
Klee e Minty apresentaram um politopo especialmente projetado para que o m�todo simplex leve um n�mero exponenencial de passos, o politopo �
denominado de Cubo de Kleen-minty.

%\subsubsection{M�todo dos elipsoides}

Em decorr�ncia da descoberta do cubo de Klee-Mint, diversos pesquisadores iniciaram um estudo em busca de um m�todo capaz de resolver o problema
de programa��o linear em tempo polinomial. Um dos primeiros trabalhos apresentados na literatura propondo um algoritmo polinomial foi o m�todo
dos elipsoides~\cite{khachian}. O m�todo consiste em criar um elipsoide que englobe a solu��o ot�ma e reduzilo-lo sequencialmente de modo que
a solu��o �tima sempre esteja dentro do elipsoide. O m�todo possui, em teoria, convergencia garantida em tempo polinomial. No entanto, na pr�tica
o m�todo possui um desempenho inferior ao simplex e apresenta problemas de instabilidade num�rica. O m�todo dos elipisoides demonstrou que diversos
problemas podem ser resolvidos em tempo polinomial~\cite{Grotschel1981}.

%\subsubsection{M�todo do ponto interior}

Em 1984 foi proposto um novo m�todo polinomial para a solu��o do problema da programa��o linear, denominado de m�todo do ponto interior.
O m�todo consiste em a partir de uma solu��o fact�vel centro do politopo, perturba-la at� que a mesma convirja para o ponto �timo. O m�todo
do ponto interior tamb�m � referenciado na literatura como m�todo das barreiras, j� que as restri��es lineares s�o reescritas como fun��es
assint�ticas que tendem ao infinito quando a restri��o linear original � violada. O m�todo do ponto interior apresentou um desempenho compar�vel
ao do simplex, e � especialmente aplicado em problemas de larga escala.


\subsection{Dualidade}

Considerando o problema de programa��o linear apresentado em~\eqref{funcao_obj},~\eqref{restricoes} e~\eqref{restricoes_triviais}, que
a partir de agora ser� denominado de problema primal,

\begin{align}
    \text{maximizar} \: z = \sum_{j=1}^{p} c_jx_j,
    \intertext{sujeito a:}
    \label{rp} \sum_{j=1}^{p} a_{ij} x_j \leq b_i, i = 1, 2, \ldots, q \\
    \label{} x_j \geq 0, j = 1, 2, \ldots, p,
\end{align}

o problema dual � constru�do atribuindo-se uma vari�vel $u_i, i = 1, 2, ..\ldots, q$ a cada restri��o em~\eqref{rp} e definindo o problema como:

\begin{align}
    \text{minimizar} \: d = \sum_{i=1}^{q} b_iu_i,
    \intertext{sujeito a:}
    \label{} \sum_{i=1}^{q} a_{ij} u_i \leq c_j, i = 1, 2, \ldots, p \\
    \label{} u_i \geq 0, i = 1, 2, \ldots, q,
\end{align}

O problema primal e dual em sua forma matricial s�o:

\begin{align}
    \label{} \text{maximizar} \: z &= cx
    \intertext{sujeito a:}
    \label{prnt} Ax &= b \\
    \label{prt} x &\geq 0,
    \intertext{e}
    \label{} \text{minimizar} \: d &= ub
    \intertext{sujeito a:}
    \label{drnt} uA &= c \\
    \label{drt} u &\geq 0,
\end{align}

A partir do problema primal e dual, conforme demonstrado em~\cite{maculan2006otimizaccao}, segue que:

\begin{enumerate}
    \label{x} \item Se $\overline{x}$ satisfaz~\eqref{prnt} e~\eqref{prt} e $\overline{u}$ satisfaz~\eqref{drnt} e~\eqref{drt}, ent�o $c\overline{x} \leq \overline{u}b$;
    \item Se $\overline{x}$ e $\overline{u}$ forem solu��es fact�veis do problema primal e dual, respectivamente, e $c\overline{x} = \overline{u}b$, ent�o
$\overline{x}$ � a solu��o �tima do problema primal e $\overline{u}$ � a solu��o �tima do problema dual;
    %\item Se o problema primal tiver seu espa�o de busca ilimitado, ent�o o problema dual � infact�vel, e vice-versa;
    \item Se $\tilde{x}$ � a solu��o �tima do problema primal e $\tilde{u}$ � a solu��o �tima do problema dual, ent�o $c\tilde{x} = \tilde{u}b$.
\end{enumerate}

O primeiro item � conhecido como teorema da dualidade fraca, e sua implica��o � que uma solu��o dual � um limite superior de otimalidade para o problema primal.
O segundo e terceiro item constituem o teorema da dualidade forte, que pode ser utilizado para provar que uma solu��o de um problema primal � a �tima.
Segue ainda que apenas uma das tr�s alternativas abaixo � verdadeira para um problema primal e o seu problema dual associado:

\begin{itemize}
    \item Ambos os problems tem um espa�o de busca vazio;
    \item Um deles tem o seu espa�o de busca vazio e o outro ilimitado;
    \item Ambos possuem o mesmo valor na fun��o objetivo e possuem solu��es �timas finitas;
\end{itemize}

\section{Programa��o Linear Inteira}
texto

\section{Gera��o de Colunas}
texto


\chapter{Considera\c{c}\~oes finais}

Neste trabalho apresentou-se a fundamenta��o te�rica referente a programa��o linear e a programa��o linear inteira,
com a finalidade de definir o problema de programa��o de tripula��o, do ingl�s \textit{crew scheduling problem} (CSP).
Apresentou-se tamb�m o m�todo de gera��o de colunas, utilizado para resolver problemas com um grande n�mero
de vari�veis, que � o caso do CSP. Modelou-se os dois componentes da gera��o de colunas para o CSP: O problema
mestre e o subproblema. Apresentou-se ainda um conjunto de m�todos para resolver o CSP juntamente com o
algoritmo h�brido exato proposto.

Durante o desenvolvimento deste trabalho pode-se identificar os principais conceitos necess�rios para compreender
o processo de solu��o do CSP. Pode-se ainda identificar um problema de interesse pr�tico e utilizado em larga
escala no setor de transporte terrestre e a�reo. Com base na utiliza��o na literatura e ao desempenho obtido
em testes de laborat�rio, optou-se por utilizar o CPLEX para resolver o problema mestre e o sub problema de modo
exato e o SCIP para realizar o \textit{Branch and Price}.

Os resultados encontrados apontam que a adi��o de heur�sticas que sejam capazes de encontrar algumas poucas colunas j� � o suficiente
para reduzir o tempo computacional total de processamento.
Identificou-se ainda atrav�s da an�lise da taxa de converg�ncia, do total de colunas geradas, do tempo computacional total e da taxa de acerto que:
Quanto maior a taxa de acerto mais r�pida e a converg�ncia e menor o tempo necess�rio para obter-se a solu��o �tima do problema; A adi��o
de colunas em excesso tem um impacto negativo na solu��o do subproblema de modo exato, que � necess�rio para garantir que a solu��o �tima foi encontrada.

Em trabalhos futuros, sugere-se analisar o algoritmo com inst�ncias reais e encorporar novas heur�sticas, j� que a adi��o de novas heur�sticas
aponta que existe vantagem em sua utiliza��o. A encorpora��o da regra de \textit{branching} de~\cite{ryan1981integer}, segundo a literatura encontrada, pode
trazer benef�cios se implementada dentro do \textit{branch and price}, devido a sua capacidade de gerar uma �rvore mais balanceada do que
o \textit{branching} de vari�vel.

O trabalho desenvolvido neste TCC faz parte de um projeto de Inicia��o Cient�fica (IC), na qual foi desenvolvido
e publicado artigos no Simp�sio Brasileiro de Pesquisa Operacional (SBPO).


\bibliographystyle{abnt-alf}
\bibliography{../4_pos_texto/bibliografia}

\annex
\chapter{Anexo}
\renewcommand{\thesection}{\Roman{section}}
\section{Decreto 3.298  de 1999, artigo 19}
\label{decreto1}
Decreto 3.298  de 1999, artigo 19:
\begin{quotation}``Consideram-se ajudas t�cnicas, para os efeitos deste Decreto, os elementos que permitem 
compensar uma ou mais limita��es funcionais motoras, sensoriais ou mentais da pessoa 
portadora de defici�ncia, com o objetivo de permitir-lhe superar as barreiras da comunica��o e da 
mobilidade e de possibilitar sua plena inclus�o social. 
Par�grafo �nico. S�o ajudas t�cnicas: 
\renewcommand{\theenumi}{\Roman{enumi}}
\begin{enumerate}
\item pr�teses auditivas, visuais e f�sicas; 
\item �rteses que favore�am a adequa��o funcional; 
\item equipamentos e elementos necess�rios � terapia e reabilita��o da pessoa portadora de 
defici�ncia; 
\item equipamentos, maquinarias e utens�lios de trabalho especialmente desenhados ou 
adaptados para uso por pessoa portadora de defici�ncia; 
\item elementos de mobilidade, cuidado e higiene pessoal necess�rios para facilitar a 
autonomia e a seguran�a da pessoa portadora de defici�ncia; 
\item elementos especiais para facilitar a comunica��o, a informa��o e a sinaliza��o para 
pessoa portadora de defici�ncia; 
\item equipamentos e material pedag�gico especial para educa��o, capacita��o e recrea��o 
da pessoa portadora de defici�ncia; 
\item adapta��es ambientais e outras que garantam o acesso, a melhoria funcional e a 
autonomia pessoal; e 
\item bolsas coletoras para os portadores de ostomia."
\end{enumerate}
 \end{quotation}
\newpage
\section{ISO 9999:2011}
\label{iso9999}

\begin{figure}[bth!]
  \begin{center}
      \centering
      \includegraphics[page=1,scale=0.7]{../artigos/iso9999.pdf}\\
    
    \label{fig:arvore}
  \end{center}
\end{figure}
\newpage
\section{Termo de Consentimento Livre e Esclarecido}
\label{apendice2}
\includegraphics[scale=0.8]{../figuras/termo.pdf}

\appendix
\chapter{Ap�ndice}
\renewcommand{\thesection}{\Alph{section}}
\section{Quadros Cl�nicos de Paralisia Cerebral}
\label{apendice1}
\begin{enumerate}
\item {Hemiplegia : � a manifesta��o mais frequente, com
maior comprometimento do membro superior;
acompanha-se de sinais de libera��o tais como
espasticidade , hiper reflexia e sinal de Babinski. O
paciente assume atitude em semiflex�o do membro
superior, permanecendo o membro inferior
hiperestendido e aduzido, e o p� em postura equinovara.
� comum hipotrofia dos segmentos acometidos, sendo
tamb�m poss�vel a ocorr�ncia de outras hemi-hipoestesia
ou hemianopsia.}
\item{ Hemiplegia bilateral ( tetra ou quadriplegia) :
Ocorrem de 9 a 43\% dos pacientes. Ocorrem les�es
difusas bilateral no sistema piramidal dando al�m da
grave tetraparesia esp�stica com intensas retra��es em
semiflex�o, s�ndrome pseudobulbar (hipomimia, disfagia
e disartria), podendo ocorrer ainda microcefalia,
defici�ncia mental e epilepsia.}
\item{Diplegia : Ocorre em 10 a 30 \% dos pacientes, sendo
a forma mais encontrada em prematuros. Tratase de um
comprometimento dos membros inferiores, comumente
evidenciando uma acentuada hipertonia dos adutores,
que configura em alguns doentes o aspecto semiol�gico
denominado s�ndrome de Little (postura com cruzamento
dos membros inferiores e marcha em tesoura). H� 
diferentes grada��es quanto � intensidade do dist�rbio,
podendo ser pouco afetado (tendo recupera��o e bom
progn�stico  adaptam-se � vida di�ria); enquanto outros
evoluem mal com graves limita��es funcionais. Os dados
semiol�gicos s�o muito vari�veis. No 1� ano de vida, a
crian�a apresenta-se hipot�nica, evoluindo
gradativamente para uma outra fase em que se observa
um quadro de distonia intermitente, com tend�ncia ao
opist�tono quando estimulada. Nos casos mais graves a
crian�a pode permanecer num destes est�gios por toda
a sua vida, por�m geralmente passa a exibir hipertonia
esp�stica, inicialmente extensora e, finalmente, com
graves retra��es semiflexoras.}
\item{Discinesia : Atualmente � a mais rara, pois
manifesta-se atrav�s de movimentos involunt�rios,
sobretudo distonias axiais e/ou movimentos c�reoatet�ides
das extremidades. No primeiro ano de vida este
padr�o costuma n�o estar definido, podendo existir
hipotonia muscular. Em geral, quando estes pacientes
est�o relaxados a movimenta��o passiva � facilitada.}
\item{Ataxia : Igualmente rara. Inicialmente pode traduzir se
por hipotonia e, aos poucos, verificam-se altera��es
do equil�brio (ataxia axial) e, menos comumente, da
coordena��o ( ataxia apendicular). Sua marcha se faz com
aumento da base de sustenta��o podendo apresentar
tremor intencional.}
\item{Formas mistas : � a associa��o das manifesta��es
anteriores, correspondendo, geralmente, ao encontro de
movimentos dist�nicos e c�reo atet�ides ou �
combina��o de ataxia com plegia (sobretudo diplegia).
No total, cerca de 75\% dos pacientes doentes com
paralisia cerebral apresentam padr�o esp�stico.
Al�m do dist�rbio motor, obrigat�rio para a
caracteriza��o da paralisia cerebral, o quadro cl�nico pode
incluir tamb�m outras manifesta��es acess�rias com
frequ�ncia vari�vel: 
\begin{enumerate}
\item{ Defici�ncia mental: Ocorre de 30 a
70\% dos pacientes. Est� mais associada �s formas
tetrapl�gicas, dipl�gicas ou mistas;}
\item{ Epilepsia: Varia de
25 a 35\% dos casos, ocorrendo mais associado com a
forma hemipl�gica ou tetrapl�gica; }
\item{ Dist�rbios da
linguagem; }
\item{ Dist�rbios visuais : Pode ocorrer perda da
acuidade visual ou dos movimentos oculares
\(estrabismo\); }
\item{ Dist�rbios do comportamento : S�o mais
comuns nas crian�as com intelig�ncia normal ou lim�trofe,
que se sentem frustradas pela sua limita��o motora,
quadro agravado em alguns casos pela super prote��o
ou rejei��o familiar; e}
\item{ Dist�rbios ortop�dicos : Mesmo
nos pacientes submetidos � reabilita��o bem orientada,
s�o comuns retra��es fibrotend�neas \(50\%\) cifoescoliose
\(15\%\), coxa valga \(5\%\) e deformidades nos p�s.
Todos esses dist�rbios se d�o devido a altera��es
nas �reas motoras cerebrais espec�ficas durante a
inf�ncia.}
\end{enumerate}
}
\end{enumerate}
\newpage
\section{Entrevista}
\label{entrevista}
\includegraphics[scale=0.8]{../figuras/entrevista.jpg}


\end{document}
