Assistive Technology (AT) is a knowledge field used to identify resources in order to provide or expand skills of people with disabilities, mobility disability or reduced mobility. 
%
This work initially presents a bibliography research about people with disabilities, specifically people who have Cerebral Palsy (CP) followed by an applied research in order to specify the requirements and project for a Extended Alternative Communication software solution. 
%
Based on the set of people with CP, this work addresses specifically the people who have limited locomotor skills in addition to speech difficulties. 
%
The alternative solution software specification presented in this work allows these people to communicate with their therapists in order to stimulate their cognition. 
%
This paper is organized as follows: context, research results related to people with disabilities; AT initiatives and their taxonomies, the specification of the problem that is addressed in the work; and specification of the proposal suggested by the author. 
%
However, the AT initiatives tend to have some regionalism directed related to local demands. 
%
Thus, this work also includes interactions with the Associa��o dos Deficientes de Joinville (ADEJ) in order to identify this regional context and the Grupo Assitiva of the Santa Catarina State University (UDESC).



%Assistive Technology are an area of knowledge that is use to identify the resources to provide or amplify skills of people with any kind of deficiency, disabilities or reduced mobility. In this work will be done a referency research, about people that have deficiency specifically people that have Cerebral Palsy, followed by a applied research to spec and develop a Alternative Comunication software. Inside the group of cerebral palsy people, the work threats specifically, who have speech deficiency and reduced mobility. The work present a alternative solution to this people communicate with their therapist and improve their cognition. The work is divided in three parts: the introduction, that says about people with general deficiencys, the Assistive Technology iniciatives and your classifications, the specification of the problem, that will be done in the work and the specification of the autor proposal.
