\chapter{Considera\c{c}\~oes finais}

%Neste trabalho apresentou-se a fundamenta��o te�rica referente a programa��o linear e a programa��o linear inteira,
%com a finalidade de definir o problema de aloca��o de tripula��o, do ingl�s \textit{crew scheduling problem} (CSP).
%Apresentou-se tamb�m o m�todo de gera��o de colunas, utilizado para resolver problemas com um grande n�mero
%de vari�veis, que � o caso do CSP. Modelou-se os dois componentes da gera��o de colunas para o CSP: O problema
%mestre e o subproblema. Por fim apresentou-se um conjunto de m�todos para resolver o CSP juntamente com a proposta
%deste trabalho.

%Durante o desenvolvimento deste trabalho pode-se identificar os principais conceitos necess�rios para compreender
%o processo de solu��o do CSP. Pode-se ainda identificar um problema de interesse pr�tico e utilizado em larga
%escala no setor de transporte terrestre e a�reo. Com base na utiliza��o na literatura e ao desempenho obtido
%em testes de laborat�rio, optou-se por utilizar o CPLEX para resolver o problema mestre e o sub problema de modo
%exato.

%A pr�xima etapa do trabalho consiste em implementar o m�todo de gera��o de colunas, validar a implementa��o,
%implementar as diferentes (meta) heur�sticas mencionadas na se��o\ref{secc4}, executar testes e coletar dados, e
%por fim fazer uma an�lise dos resultados encontrados. Tem-se como objetivo tamb�m escrever um artigo apresentando
%os resultados obtidos neste trabalho.

%O trabalho desenvolvido neste TCC faz parte de um projeto de Inicia��o Cient�fica (IC), na qual foi desenvolvido
%e publicado um artigo no Simp�sio Brasileiro de Pesquisa Operacional (SBPO).
