\documentclass{article}

\usepackage[english, brazilian]{babel}
\usepackage[T1]{fontenc}
\usepackage[latin1]{inputenc}
\usepackage{tikz}
\usetikzlibrary{arrows}
\usetikzlibrary{calc}
\usetikzlibrary{fit}
\usetikzlibrary{positioning}
\usetikzlibrary{shapes.geometric}
\usetikzlibrary{shapes.misc}
\usetikzlibrary{shapes}

\usepackage{verbatim}
\usepackage[active,tightpage]{preview}
\PreviewEnvironment{tikzpicture}
\setlength\PreviewBorder{5pt}%

\begin{document}
\pagestyle{empty}


% Define block styles
\tikzstyle{decision} = [diamond, draw, fill=blue!20,
    text width=4.5em, text badly centered, node distance=3cm, inner sep=0pt]
\tikzstyle{block} = [rectangle, draw, fill=blue!20,
    text width=5em, text centered, rounded corners, minimum height=4em]
\tikzstyle{line} = [draw, -latex']
\tikzstyle{cloud} = [draw, ellipse,fill=red!20, node distance=3cm,
    minimum height=2em]

\begin{tikzpicture}[node distance = 2cm, auto]
    \node [block,text width=25mm              ] (f1) at (0,0) {Projeto das rotas};
    \node [block,text width=25mm, below of=f1 ] (f2)          {Planejamento dos hor\'arios};
    \node [block,text width=25mm, below of=f2 ] (f3)          {Escalonamento de Ve\'iculos};
    \node [block,text width=25mm, below of=f3 ] (f4)          {Escalonamento de Tripula\c{c}\~ao};

    \path [line] (f1) -> (f2);
    \path [line] (f2) -> (f3);
    \path [line] (f3) -> (f4);
\end{tikzpicture}


\end{document}
